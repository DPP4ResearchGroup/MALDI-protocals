\textbf{R&D Proteome Purify} 
\\
\textsc{Removal of albumin and IgG}
\\
\textbf{Composition:} 
\begin{enumerate}
    \item{50/50 (v/v) slurry of proprietary absorption gel in PBS with 0.02\% sodium azide (pH7.4)}
    \item{Tube filter with 0.22 \mu m cellulose acetate membrane (capable of 1K-2K x g)}
\end{enumerate}

\textbf{Material:}
\\
\begin{enumerate}
    \item {Microcentrifuge 1K -2K X g}
    \item {Rotary shaker or mixer}
    \item {Microcentrifuge tubes 10 M}
    \item {Test tubes 16 x 100 mm}
\end{enumerate}

\textbf{Procedures}
\\
\textit{Bring immunodepletion Resin to room temp before use and perform at room temp} \\
\textit{It is essential that the immunodepletion Resin be homogeneous. Mix well bey inversion prior to pipetting}
\begin{enumerate}
    \item {Add 10 \mu L of serum or plasma to a test tube}
    \item {Add 1.0 mL of suspended Immunodepletion Resin to the test tube containing the sample}
    \item {Place the tube on a rotary shaker and mix for 30-60 mins. The mix speed should be sufficient to keep the immunodepletion Resin in suspension}
    \item {After the incubation period, pepette equal amounts of the Immunodepletion Resin Mixture into the upper chamber of two Spin-X Filter}
    \item {Centrifuge for 2 minutes at 1K-2K x g in the microcentrifuge tube. The volume of the combined filtrate will be approximately 400 - 500 \mu L.}
    \ item {The sample is now prepared and ready for storage or down stream analysis and discard the used immunodepletion Resin}
\end{enumerate}
\footnotesize
\textit{Note: Initial depletion of the serum will result in the removal of greater than 90\% of the two high abundant proteins. If higher depletion are required, the sample should be concentrated (5000 Da MWCO spin concentrator) and depleted a scond time}
